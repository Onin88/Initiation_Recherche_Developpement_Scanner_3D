% Page 11
\thispagestyle{empty} % Supprimer l'en-tête et le pied de page
\begin{center}
    \section{\huge\textbf{{GLOSSAIRE}}}

\end{center}
\textbf{Chaque mot compliqué sera expliqué de manière général et par rapport à notre projet.} \\

\paragraph{$^1$ Densité\label{def:densite}}: La densité se rapporte à la concentration ou au nombre de points dans un nuage de points 3D. Une densité élevée signifierait qu'il y a beaucoup de points dans un espace donné, ce qui peut permettre une représentation plus détaillée d'une surface ou d'un objet.

\paragraph{$^2$ Stéréoscopique\label{def:stereoscopique}}: La stéréoscopie se rapporte à une méthode de capture ou de représentation qui implique deux images (ou plus) prises de points de vue légèrement différents, simulant ainsi la perception de la profondeur

\paragraph{$^3$ Calibration\label{def:calibration}}: La calibration fait référence au processus de réglage et de standardisation des paramètres d'un dispositif de capture pour garantir des mesures précises et cohérentes, dans notre cas ajuster les paramètres du Lidar et de la Zed2.

\paragraph{$^4$ Post-traitement\label{def:post_traitement}}: Le post-traitement se réfère aux opérations effectuées sur les données après leur acquisition initiale. Traitement avant utiliastion

\paragraph{$^5$ Faisceaux laser\label{def:faisceau_laser}}: Les faisceaux laser sont des faisceaux lumineux concentrés et cohérents qui peuvent être utilisés pour mesurer des distances avec une grande précision. Dans le contexte du lidar, les faisceaux laser sont émis par le dispositif pour mesurer la distance aux objets environnants, permettant ainsi la création de nuages de points 3D.

\paragraph{$^6$ Matrice de projection\label{def:matric_proj}}: En vision par ordinateur et en traitement d'image, une matrice de projection est une matrice mathématique utilisée pour transformer des points dans un espace vers un autre espace. Cette transformation est essentielle pour projeter les nuages de points 3D sur une image.

\paragraph{$^7$ Traitement matriciel\label{def:traitement_matriciel}}: Le traitement matriciel fait référence à l'utilisation de techniques mathématiques sur des matrices pour effectuer des opérations telles que la multiplication, l'inversion, la factorisation, etc. Dans notre projet, le traitement matriciel est utilisé pour effectuer des multiplications de matrices.

\paragraph{$^8$ Robustesse\label{def:robustesse}}:  La robustesse se rapporte à la capacité d'un système ou d'un dispositif à maintenir sa performance dans des conditions variables ou difficiles. Dans notre contexte, la robustesse des technologies de capture se réfère à leur capacité à fonctionner de manière fiable dans différents environnements et conditions d'éclairage.  

\clearpage