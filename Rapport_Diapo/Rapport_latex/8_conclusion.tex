% Page 8
\thispagestyle{empty} % Supprimer l'en-tête et le pied de page
\begin{center}
    \section{\huge\textbf{{CONCLUSION}}}
\end{center}

\normalsize{
    En conclusion, notre travail a été guidé par une exploration minutieuse des différentes technologies de capture de données 3D, avec un objectif clair de colorier un nuage de points à partir d'un flux d'images. À travers nos évaluations détaillées de la caméra Zed2, de la Kinect v1 et du Lidar Velodyne Puck Lite, nous avons pu identifier les forces et les limitations de chaque solution. Bien que la Zed2 offre une polyvalence documentée, nous avons constaté que sa performance en matière de traitement des nuages de points était moins satisfaisante que celle de la Kinect v1, qui a été une agréable surprise en termes de précision et de qualité des résultats. Le Lidar Puck Lite, quant à lui, s'est démarqué par sa rapidité de traitement et sa capacité à fournir des nuages de points denses et précis.
}
\\ \\
\normalsize{
    Ce projet nous a permis d'acquérir une compréhension approfondie des technologies de capture 3D et de développer des compétences pratiques dans le traitement et la manipulation de données volumineuses. Nous avons également appris à surmonter des défis techniques et à nous adapter rapidement à de nouveaux concepts et technologies, renforçant ainsi notre capacité à aborder des projets complexes à l'avenir.
}
\\ \\
\normalsize{
    En termes de perspectives futures, notre expérience nous a incités à envisager des développements supplémentaires dans le domaine de la capture et du traitement de données 3D, en particulier en explorant de nouvelles technologies émergentes et en affinant nos méthodes de traitement des nuages de points. Enfin, nous sommes convaincus que les enseignements tirés de cette expérience nous serviront de fondement solide pour aborder de futurs projets avec confiance et compétence.
}

\clearpage