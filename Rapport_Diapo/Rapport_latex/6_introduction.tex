% Page 5
\thispagestyle{empty} % Supprimer l'en-tête et le pied de page
\begin{center}
    \section{\huge\textbf{{INTRODUCTION}}}  
\end{center}

\subsection{Contexte}
\normalsize{
    \setlength{\parindent}{1cm} % Définir l'alinéa à 1cm à l'intérieur du groupe
    Au sein de l'équipe SIMBIOT, encadrés par S. Contassot-Vivier et A. Saint-Jore, nous allons nous initier dans le domaine de la robotique grâce à un projet intitulé "Développement d'un scanner 3D dense avec couleurs". Notre mission consiste à développer un système de capture générant des nuages denses de points colorés, offrant ainsi un large éventail de possibilités d'application, allant de la cartographie en temps réel à la surveillance environnementale. Cette initiative s'inscrit dans le domaine de l'exploration et de la navigation autonome par robots mobiles, pour par exemple servir un autre projet qui permettra à un chien robot de naviguer dans des grottes, où l'environnement est dense, dangereux, instable et obscure.
}


\\
\subsection{Problèmatique}

\normalsize{
    \setlength{\parindent}{1cm} % Définir l'alinéa à 1cm à l'intérieur du groupe
    L'un des défis majeurs rencontrés dans ce domaine est le développement d'un système de capture capable de générer des nuages denses de points colorés avec une précision et une fidélité optimales. Ces nuages de points constituent une représentation tridimensionnelle détaillée de l'environnement immédiat du robot, fournissant des informations cruciales pour la navigation, la planification de trajectoire et la perception d'objets. Plusieurs questions s'offrent à nous, quelle technologie serait la mieux adaptée à ce projet ? Quelles sont les capacités, le rendu et les différences entre chacune ? Est-ce possible d'avoir un rendu optimal d'un nuage de point colorés?
}

\\
\subsection{Plan}


\normalsize{
    \setlength{\parindent}{1cm} % Définir l'alinéa à 1cm à l'intérieur du groupe
    Pour répondre à cette problématique, notre rapport est structuré autour de plusieurs axes. Tout d'abord, nous procéderons à une étude bibliographique approfondie sur les techniques de scan 3D et de coloriage de points 3D, afin de découvrir ce monde tridimensionnel. Ensuite, nous détaillerons les différentes étapes de nos expérimentations, depuis l'installation et la configuration des équipements de capture jusqu'à la mise en œuvre d'algorithmes de traitement des données et l'évaluation des résultats obtenus. Enfin, nous discuterons des perspectives futures pour notre recherche, en identifiant les défis restants et les pistes d'amélioration envisageables pour faire progresser les technologies de perception pour les robots mobiles autonomes.
}

\clearpage