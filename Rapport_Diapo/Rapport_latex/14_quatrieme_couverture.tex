% Page 4
\thispagestyle{empty} % Supprimer l'en-tête et le pied de page
\begin{center}
    \section{\huge\textbf{{RESUME FRANCAIS ET ANGLAIS}}}
\end{center}

\subsection{Francais}
    \normalsize{
        Le projet vise à évaluer différentes technologies de capture 3D en vue de développer un système capable de générer des nuages denses de points colorés. Pour se faire, nous avons suivi une méthodologie comprenant trois étapes principales : la sélection des technologies à évaluer, la réalisation de tests et d'expérimentations, et enfin l'analyse des résultats. Nous avons eu la caméra Zed2, la Kinect v1 et le Lidar Velodyne Puck Lite comme technologies à évaluer en fonction de critères tels que la précision, la vitesse de traitement et la qualité des nuages de points générés. Enfin, nous avons analysé les résultats obtenus pour identifier les forces et les faiblesses de chaque technologie et déterminer leur pertinence par rapport à notre objectif de générer des nuages de points colorés denses.  La caméra Zed2 offre une polyvalence documentée mais présente des limitations dans le traitement des nuages de points. La Kinect v1 surprend par sa précision tandis que le Lidar Puck Lite se distingue par sa rapidité de traitement et sa précision. Enfin, nous avons réussi à générer des nuages de points, et à en colorer dans les data set de Kitti grâce aux informations spatiales manquantes qu'ils nous fournissent. Les principales recommandations proposées sont de bien se documenter sur le sujet, sur la technologie et les possibilités de chaque outils mis à notre disposition.
    }

\subsection{Anglais}
    \normalsize{
        The project aims to evaluate different 3D capture technologies in order to develop a system capable of generating dense colored point clouds. To achieve this, we followed a methodology comprising three main steps: selecting the technologies to evaluate, conducting tests and experiments, and finally analyzing the results. We evaluated the Zed2 camera, the Kinect v1, and the Lidar Velodyne Puck Lite based on criteria such as precision, processing speed, and quality of the generated point clouds. Ultimately, we analyzed the results to identify the strengths and weaknesses of each technology and determine their relevance to our goal of generating dense colored point clouds. The Zed2 camera offers well-documented versatility but has limitations in processing point clouds. The Kinect v1 impresses with its precision, while the Lidar Puck Lite stands out for its processing speed and precision. Finally, we succeeded in generating point clouds and coloring them in the Kitti data set using the spatial information they provide. The main recommendations proposed are to thoroughly research the subject, technology, and capabilities of each tool at our disposal.
    }

\clearpage